% Copyright 2018 Melvin Eloy Irizarry-Gelpí
\chapter{Reals}
%%%%%%%%%%%%%%%%%%%%%%%%%%%%%%%%%%%%%%%%%%%%%%%%%%%%%%%%%%%%%%%%%%%%%%%%%%%%%%%%
The real numbers are the set of familiar numbers. There are integers, rationals, and irrationals.
%%%%%%%%%%%%%%%%%%%%%%%%%%%%%%%%%%%%%%%%%%%%%%%%%%%%%%%%%%%%%%%%%%%%%%%%%%%%%%%%
\section{Arithmetic Operations}
%%%%%%%%%%%%%%%%%%%%%%%%%%%%%%%%%%%%%%%%%%%%%%%%%%%%%%%%%%%%%%%%%%%%%%%%%%%%%%%%
For future reference, here you can find the familiar arithmetic operations on real numbers.
%%%%%%%%%%%%%%%%%%%%%%%%%%%%%%%%%%%%%%%%%%%%%%%%%%%%%%%%%%%%%%%%%%%%%%%%%%%%%%%%
\subsection{Addition}
%%%%%%%%%%%%%%%%%%%%%%%%%%%%%%%%%%%%%%%%%%%%%%%%%%%%%%%%%%%%%%%%%%%%%%%%%%%%%%%%
The addition operation is familiar. If $a$ and $b$ are two real numbers, then the result $c$ of the addition of $a$ and $b$ is also a real number:
\begin{equation}
    c = a + b
\end{equation}
There is an element $0$ in the reals called zero that does not contribute to the addition:
\begin{equation}
    a = a + 0 = 0 + a
\end{equation}
Note that zero does not have a sign.
%%%%%%%%%%%%%%%%%%%%%%%%%%%%%%%%%%%%%%%%%%%%%%%%%%%%%%%%%%%%%%%%%%%%%%%%%%%%%%%%
\subsection{Multiplication}
%%%%%%%%%%%%%%%%%%%%%%%%%%%%%%%%%%%%%%%%%%%%%%%%%%%%%%%%%%%%%%%%%%%%%%%%%%%%%%%%
The multiplication operation is also familiar. If $a$ and $b$ are two real numbers, then the result $c$ of the multiplication of $a$ and $b$ is also a real number:
\begin{equation}
    c = a \wedge b
\end{equation}
There is an element $1$ in the reals called one that does not contribute to the multiplication:
\begin{equation}
    a = a \wedge 1 = 1 \wedge a
\end{equation}
Note that one is positive.
%%%%%%%%%%%%%%%%%%%%%%%%%%%%%%%%%%%%%%%%%%%%%%%%%%%%%%%%%%%%%%%%%%%%%%%%%%%%%%%%
\subsection{Negation}
%%%%%%%%%%%%%%%%%%%%%%%%%%%%%%%%%%%%%%%%%%%%%%%%%%%%%%%%%%%%%%%%%%%%%%%%%%%%%%%%
There is an element $-1$ in the reals that when multiplied with a real number $a$ gives the negative of $a$:
\begin{equation}
    {-a} = a \wedge {-1} = {-1} \wedge a
\end{equation}
Note that one is negative. This operation is a 2-involution.
%%%%%%%%%%%%%%%%%%%%%%%%%%%%%%%%%%%%%%%%%%%%%%%%%%%%%%%%%%%%%%%%%%%%%%%%%%%%%%%%
\section{Conjugation Operations}
%%%%%%%%%%%%%%%%%%%%%%%%%%%%%%%%%%%%%%%%%%%%%%%%%%%%%%%%%%%%%%%%%%%%%%%%%%%%%%%%
The conjugation operations on real numbers are trivial, but are needed for Cayley-Dickson construct where they lead to non-trivial conjugations.
%%%%%%%%%%%%%%%%%%%%%%%%%%%%%%%%%%%%%%%%%%%%%%%%%%%%%%%%%%%%%%%%%%%%%%%%%%%%%%%%
\subsection{Asterisk Conjugation}
%%%%%%%%%%%%%%%%%%%%%%%%%%%%%%%%%%%%%%%%%%%%%%%%%%%%%%%%%%%%%%%%%%%%%%%%%%%%%%%%
The asterisk conjugate operation on a real number $a$ is itself:
\begin{equation}
    a^{\ast} = a
\end{equation}
That is, all real numbers are asterisk-self-conjugate. This operation is trivially a 2-involution.
%%%%%%%%%%%%%%%%%%%%%%%%%%%%%%%%%%%%%%%%%%%%%%%%%%%%%%%%%%%%%%%%%%%%%%%%%%%%%%%%
\subsection{Quadrance}
%%%%%%%%%%%%%%%%%%%%%%%%%%%%%%%%%%%%%%%%%%%%%%%%%%%%%%%%%%%%%%%%%%%%%%%%%%%%%%%%
The quadrance of a real number $a$ is defined as the product of $a$ and the asterisk conjugate of $a$:
\begin{equation}
    \Vert a \Vert^{2} = a^{2} = a \wedge a^{\ast} = a^{\ast} \wedge a
\end{equation}
Note that the quadrance of any non-zero real number is positive, and that zero is the only real number with zero quadrance.
%%%%%%%%%%%%%%%%%%%%%%%%%%%%%%%%%%%%%%%%%%%%%%%%%%%%%%%%%%%%%%%%%%%%%%%%%%%%%%%%
\subsection{Cloak Conjugation}
%%%%%%%%%%%%%%%%%%%%%%%%%%%%%%%%%%%%%%%%%%%%%%%%%%%%%%%%%%%%%%%%%%%%%%%%%%%%%%%%
The cloak conjugate operation on a real number $a$ is equivalent to the negation operation:
\begin{equation}
    a^{\diamond} = {-a}
\end{equation}
This operation is a 2-involution.
%%%%%%%%%%%%%%%%%%%%%%%%%%%%%%%%%%%%%%%%%%%%%%%%%%%%%%%%%%%%%%%%%%%%%%%%%%%%%%%%
\subsection{Dagger Conjugation}
%%%%%%%%%%%%%%%%%%%%%%%%%%%%%%%%%%%%%%%%%%%%%%%%%%%%%%%%%%%%%%%%%%%%%%%%%%%%%%%%
The dagger conjugate operation on a real number $a$ is itself:
\begin{equation}
    a^{\dagger} = a
\end{equation}
That is, all real numbers are dagger-self-conjugate. This operation is trivially a 2-involution.
%%%%%%%%%%%%%%%%%%%%%%%%%%%%%%%%%%%%%%%%%%%%%%%%%%%%%%%%%%%%%%%%%%%%%%%%%%%%%%%%
\subsection{Hodge Star}
%%%%%%%%%%%%%%%%%%%%%%%%%%%%%%%%%%%%%%%%%%%%%%%%%%%%%%%%%%%%%%%%%%%%%%%%%%%%%%%%
The Hodge star operation on a real number $a$ is itself:
\begin{equation}
    a^{\star} = a
\end{equation}
That is, all real numbers are star-self-conjugate. This operation is trivially a 2-involution.
%%%%%%%%%%%%%%%%%%%%%%%%%%%%%%%%%%%%%%%%%%%%%%%%%%%%%%%%%%%%%%%%%%%%%%%%%%%%%%%%
\section{Differential Operators}
%%%%%%%%%%%%%%%%%%%%%%%%%%%%%%%%%%%%%%%%%%%%%%%%%%%%%%%%%%%%%%%%%%%%%%%%%%%%%%%%
You can define a real variable $z$ and a \textbf{Wirtinger} differential operator $\nabla$ such that
\begin{equation}
    \nabla \wedge z = 1
\end{equation}
That is,
\begin{equation}
    \nabla = \frac{\partial}{\partial z}
\end{equation}
Since real numbers are asterisk-self-conjugate, the asterisk Wirtinger differential operator $\nabla^{\ast}$ is equivalent to $\nabla$:
\begin{equation}
    \nabla^{\ast} = \nabla
\end{equation}
The quadrance of the Wirtinger differential operator is the \textbf{Laplace} differential operator:
\begin{equation}
    \Vert \nabla \Vert^{2} = \frac{\partial^{2}}{\partial z^{2}} = \nabla \wedge \nabla^{\ast} = \nabla^{\ast} \wedge \nabla
\end{equation}